\documentclass{VUMIFPSbakalaurinis}
\usepackage{algorithmicx}
\usepackage{algorithm}
\usepackage{algpseudocode}
\usepackage{amsfonts}
\usepackage{amsmath}
\usepackage{bm}
\usepackage{caption}
\usepackage{color}
\usepackage{float}
\usepackage{graphicx}
\usepackage{listings}
\usepackage{subfig}
\usepackage{wrapfig}

% Titulinio aprašas
\university{Vilniaus universitetas}
\faculty{Matematikos ir informatikos fakultetas}
\department{Programų sistemų katedra}
\papertype{Bakalauro darbas}
\title{-}
\titleineng{-}
\status{4 kurso 5 grupės studentas}
\author{Džiugas Baltramėnas}
\supervisor{asist. Karolis Uosis}
\date{Vilnius – \the\year}

% Nustatymai
%\setmainfont{Time}   % Pakeisti teksto šriftą į Palemonas (turi būti įdiegtas sistemoje)
\bibliography{bibliografija}

\begin{document}
\maketitle

\tableofcontents

\sectionnonum{Įvadas}
    -

    \textbf{Temos aktualumas}: -

    \textbf{Darbo tikslas}:
    -
    \textbf{Darbo Uždaviniai}:
    \begin{enumerate}
        \item -;
        \item -;
        \item -;
        \item -.
    \end{enumerate}



\sectionnonum{Rezultatai ir išvados}
-

Atlikus darbą buvo gauti šie \textbf{rezultatai}:
\begin{enumerate}
    \item -;
    \item -;
    \item -;
    \item -.
\end{enumerate}

Darbo \textbf{išvados}:
\begin{enumerate}
    \item -;
    \item -;
    \item -;
    \item -.
\end{enumerate}

\printbibliography[heading=bibintoc]  % Šaltinių sąraše nurodoma panaudota
% literatūra, kitokie šaltiniai. Abėcėlės tvarka išdėstomi darbe panaudotų
% (cituotų, perfrazuotų ar bent paminėtų) mokslo leidinių, kitokių publikacijų
% bibliografiniai aprašai.  Šaltinių sąrašas spausdinamas iš naujo puslapio.
% Aprašai pateikiami netransliteruoti. Šaltinių sąraše negali būti tokių
% šaltinių, kurie nebuvo paminėti tekste.

\sectionnonum{Sąvokų apibrėžimai}
\begin{enumerate}
    \item -;
    \item -;
    \item -;
    \item -.
\end{enumerate}

\end{document}
