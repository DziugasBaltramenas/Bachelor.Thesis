\section{Informacinių technologijų taikymas sveikatos priežiūros įstaigose}
 Kadangi Europos sąjungos populiacija sensta, o specialistų trūksta \cite{Organization2012} \cite{Eurostat}, vadinasi reikalingas sveikatos priežiūrios procesų efektyvumo didėjimas, kad padengtų senstančios populiacijos reikmes. Šių procesų efektyvumą galime didinti informacinių technologijų pagalba. Toliau šiame skyriuje apžvelgsime informacinių technologijų taikymą Lietuvos sveikatos priežiūros įstaigose.

\subsection{Gydymo įstaigų informacinės sistemos}
Gydymo įstaigų informacinė sistema, kitaip dar vadinama hospitalinė informacinė sistema (toliau - HIS), yra autonominė sveikatos priežiūros įstaigos sistema, kuri orientuojasi į šias veiklas: pacientų registravimas, priėmimas, išleidimas, perkėlimas, apmokestinimas ir kitas administracines, finansines ir medicinines funkcijas \cite{Sabooniha2012}. Tam, kad ši sistema išties gerintų sveikatos priežiūros procesų efektyvumą, reikalingas tinkamas duomenų paskirstymas tarp sveikatos priežiūros įstaigos skyrių, todėl šios sistemos pagrindinis uždavinys - apjungti visų skyrių informacines sistemas. Pati HIS nėra laikoma individualaus skyriaus informacine sistema \cite{JuliusGriskevicius}, ji priima klinikinius duomenis iš įstaigos skyrių informacinių sistemų ir juos saugo. Sveikatos priežiūros įstaigos specialistams prireikus pacientų klinikinių duomenų, HIS suteikia prieigą prie jų. Pagrindinės savybės apibūdinančios HIS \cite{JuliusGriskevicius}: 
\begin{enumerate}
    \item Informacijos apie pacientus duomenų bazės;
    \item Pacientų priėmimas ir lovų užimtumo kontrolė;
    \item Analizavimo įrankiai, kurie palengvina sprendimo priėmimą;
    \item Pacientų valdymas ir jų sveikatos įvertinimas.
\end{enumerate}


Tam, kad suprasti Lietuvos sveikatos priežiūros įstaigų naudojamų informacinių sistemų savybes ir funkcionalumus, autorius pasirinko išanalizuoti trijų sveikatos įstaigų HIS. Buvo pasirinkta Kauno regiono asmens sveikatos priežiūros įstaigų HIS (toliau - KRASPĮ), nes šią HIS naudoja Jonavos ligoninė, ir Vilniaus universiteto ligoninės Žalgirio klinikos HIS (toliau - VULŽK HIS). Šios dvi HIS buvo pasirinktos todėl, nes įstaigos, naudojančios šias HIS, buvo pasirinktos stacionaraus gydymo nagrinėjimui ankstesniame skyriuje. Taip pat buvo pasirinkta nagrinėti Vilniaus universiteto ligoninės Santaros klinikos HIS (toliau - VULSK HIS), nes tai yra viena iš didžiausių Lietuvos sveikatos priežiūros įstaigų. Išanalizavimus šių įstaigų HIS specifikacijas, buvo sudaryta palyginimo lentelė (žiūrėti 1 lentelę). 

\begin{table}[!ht]
    \centering
    \renewcommand{\arraystretch}{1.2}
    \caption{Sveikatos priežiūros įstaigų informacinių sistemų palyginimas}
    \begin{tabular}{| p{8em} | c | c | c |}\hline
        \backslashbox[8em]{Savybės}{Įstaigos}
        &\makebox[8em]{KRASPĮ}&\makebox[8em]{VULŽK HIS}&\makebox[8em]{VULSK HIS}\\\hline
        Duomenų centralizuotas kaupimas &+ &+ &+\\\hline
        Pacientų valdymas &+ &+ &+\\\hline
        Duomenų analizavimo įrankiai &- &- &+\\\hline
        Pacientų sveikatos vertinimas &+ &+ &+\\\hline
        Finansinių dokumentų paruošimas &- &+ &+\\\hline
        Lovų užimtumo kontrolė &+ &+ &+\\\hline
    \end{tabular}
    \label{HIS}
\end{table}

Atlikus palyginimą, buvo išsiaiškinta, kad visos nagrinėjamos HIS pasižymi panašiomis savybės ir sprendžia bendras problemas. Tačiau tiek KRASPĮ HIS, tiek VULŽK HIS, trūksta duomenų analizavimo įrankių, taip pat KRASPĮ nėra finansinių dokumentų paruošimo funkcionalumo.


\subsection{Elektroninė sveikatos paslaugų ir bendradarbiavimo infrastruktūros informacinė sistema}
\subsubsection{Sistemos apžvalga}
Ankstesniame poskyryje buvo apžvelgta gydymo įstaigų informacinės sistemos, tačiau pacientams ne visada sveikatos priežiūros paslaugos teikiamos vienoje gydymo įstaigoje. Pacientui keičiant gydymo įstaigą, iškyla problema - klinikinių paciento duomenų perkėlimas. Šią problemą išsprendžia visos Lietuvos mastu naudojama elektroninė sveikatos paslaugų ir bendradarbiavimo infrastruktūros informacinė sistema (toliau - ESPBI IS). ESPBI IS yra apibrėžiama kaip priemonių visuma, kuri skirta centralizuotai kaupti, formuoti ir naudoti pacientų sveikatos istorijas \cite{ESPBINuostatos}. Šios sistemos dėka, įstaigos, kurios turi prieigos teises, gali tarpusavyje keistis pacientų informacija. Lietuvoje ESPBI IS buvo diegiama 3 etapais, diegimo datos pradžia - 2007 metai, o pabaiga - 2015 metai \cite{Ministras2015}. Kadangi ESPBI IS yra didelės apimties projektas, kuris truko beveik dešimtmetį, dalykinės srities reikalavimai buvo klasifikuojami ir kuriami mažesni projektai šiems reikalavimams įgyvendinti. Pagrindiniai ESPBI IS projektai \cite{Specifikacija}:
\begin{enumerate}
    \item \textbf{Elektroninės sveikatos paslaugų ir bendradarbiavimo infrastruktūra}.
    
    Šio projekto metu sukurtos sistemos pagrindinis funkcionalumas:
    \begin{enumerate}
        \item Pacientų elektroninio sveikatos įrašo tvarkymas;
        \item Paciento registravimas arba išregistravimas iš sveiktos priežiūros įstaigos;
        \item Sąveika tarp skirtingų sveikatos priežiūros įstaigų informacinių sistemų;
        \item Aktualių paciento duomenų teikimas ir gavimas;
        \item Finansinių ataskaitų tvarkymas;
        \item Elektroninės tapatybės nustatymas.
    \end{enumerate}
    \item \textbf{Elektroninis receptas}. 
    
    Šio projekto metu sukurtos sistemos pagrindinis funkcionalumas:
    \begin{enumerate}
        \item Elektroninių receptų ar kompensuojamų medicininės pagalbos priemonių išrašymas;
        \item Centralizuotas išrašytų receptų registravimas;
        \item Elektroninių receptų informacijos pateikimas pacientams.
    \end{enumerate}

    \item \textbf{MedVAIS}.
    
    Šio projekto metu sukurtos sistemos pagrindinis funkcionalumas:
    \begin{enumerate}
        \item Sveikatos priežiūros įstaigų sukurtų medicininių vaizdų tvarkymas medicininių vaizdų tvarkymo posistemėje;
        \item Medicininių vaizdų pateikimas pacientams;
        \item Medicininių vaizdų pateikimas gydytojams;
        \item Nuasmeninto medicininio vaizdo pateikimas;
    \end{enumerate}
\end{enumerate}

Kadangi autoriaus projektuojama sistema nėra susijusi nei su medicininiais vaizdai, nei su elektroniniais receptais, tolimesniame nagrinėjime autorius didesnį dėmesį skiria pirmojo projekto analizavimui, o sekantiems dviem projektams dėmesys skiriamas mažesnis.


ESPBI IS paskirtis yra išskiriama į paciento ir sveikatinimo įstaigų atžvilgius \cite{Specifikacija}. Paciento atžvilgiu ESPBI IS paskirtis yra:
\begin{itemize}
    \item Gerinti sveikatingumo paslaugų prieinamumą ir tęstinumą;
    \item Turėti prieigą prie sveikatą apibūdinančių dokumentų;
    \item Plėtoti elektroninės sveikatos paslaugas ir užtikrinti, kad pacientai būtų tinkamai informuojami apie teikiamas paslaugas.
\end{itemize}


Sveikatinimo įstaigų atžvilgiu ESPBI IS paskirtis yra:
\begin{itemize}
    \item Pašalinti paciento duomenų dubliavimą;
    \item Užtikrinant administracinio darbo efektyvumą;
    \item Plėtoti elektroninės sveikatos paslaugas ir užtikrinti, kad įstaigos bendradarbiautų ir gautų aktualią paciento informaciją;
    \item Užtikrinti elektroninės sveikatos paslaugų efektyvumą;
    \item Užtikrinti prieigą prie centralizuotos informacijos.
\end{itemize}

Apibendrinant išvardintas šios sistemos paskirtis, galima teigti, kad siekiama, jog pacientas turėtų galimybę peržiūrėti savo sveikatos istoriją elektroniniu būdų, o įstaigos - efektyviai keistųsi paciento informacija ir ją naudotų tam, kad paslaugų kokybė gerėtų.

\subsubsection{Sistemos architektūra}

Pagal ESPBI IS specifikaciją \cite{Specifikacija} autorius parengė abstrakčią ESPBI IS architektūros diagramą (žiūrėti 1 pav.). 

\begin{figure}[H]
    \centering
    \includegraphics[scale=0.45]{images/ESPBI}
    \caption{Elektroninės sveikatos paslaugų ir bendradarbiavimo infrastruktūros informacinės sistemos achitektūra} 
\end{figure}
% perpiesti, reikia prideti esi posisteme


Pagal parengtos architektūros diagramą matome, kad architektūra yra 3 lygmenų - vaizdavimo ir paslaugų, veiklos logikos ir duomenų lygmuo. 

\begin{itemize}
    \item \textbf{Vaizdavimo ir paslaugų lygmuo}. Šį lygmenį sudaro elektroninės sveikatos portalo posistemė. Šio portalo paskirtis - medicininių  paslaugų inicijavimas, šių paslaugų informavimas, jų vykdymo stebėjimas ir suteiktų paslaugų rezultatų pateikimas\cite{Specifikacija}. Šios sistemos naudotojai yra identifikuojami elektroniniu parašu. Prisijungusiam naudotojui yra pateikiamas turinys pagal identifikavimo metu suteiktas prieigas. Elektroninė sveikatos portalo posistemė susideda iš 4 sričių:
    \begin{enumerate}
        \item Viešai prieinama sritis. Šioje elektroninės sveikatos portalo srityje yra pateikiama visiem prieinama informacija. Tam, kad naudotojas pasiektų šią informaciją, jam savo tapatybės identifikuoti nereikia;
        \item Pacientų sritis. Šioje elektroninės sveikatos portalo srityje yra pateikiama paciento gautų paslaugų informacija, rezultatai, išrašytų elektroninių receptų duomenys ir kt;
        \item Sveikatos priežiūros specialistų sritis. Šioje elektroninės sveikatos portalo srityje sveikatos priežiūros specialistai gali pildyti elektroninės sveikatos priežiūros formas, tvarkyti pacientų duomenis, gauti aktualius paciento klinikinius duomenis, medicininius vaizdus. Taip pat specialistas, turintis reikiamą prieiga, gali išrašyti pacientui elektroninį receptą, pratęsti jo galiojimo terminą;
        \item Farmacijos specialistų sritis. Ši elektroninės sveikatos portalo sritis yra skirta farmacijos specialistams. Ji leidžia naudotojams  gauti informaciją apie išrašytą elektroninį receptą ir patvirtinti vaistų ar medicininių pagalbos priemonių išdavimo faktą.
    \end{enumerate}
    \item \textbf{Veiklos logikos lygmuo}. Šis lygmuo yra tarpinis tarp vaizdavimo ir duomenų lygmenų.  Veiklos logikos lygmens pagrindinė paskirtis yra priimti sisteminius pranešimus, juos apdoroti, sugeneruoti reikiamą informaciją ir šią informaciją pateikti vaizdavimo ir paslaugų lygmeniui \cite{Specifikacija}. Taip pat šis lygmuo priima duomenis iš elektroninio sveikatos portalo, šiuos duomenis apdoroja ir perduoda į duomenų lygmenį. Apibendrinant šio lygmens paskirtį - tai centrinis funkcionalumų lygmuo, kuris apdoroja ir pateikia reikalingus duomenis apie pacientą, elektroninius receptus, medicininius vaizdus ir kt. Šį lygmenį sudaro 11 posistemių, tačiau pagrindinės yra: 
    \begin{enumerate}
        \item Pacientų posistemė;
        \item Medicininių vaizdų posistemė;
        \item Elektroninio recepto posistemė;
        \item Duomenų analizės, ataskaitų formavimo ir informavimo posistemė;
        \item Elektroninio sveikatos įrašo posistemė.
    \end{enumerate}
    \item \textbf{Duomenų lygmuo}. Duomenų lygmuo susideda iš 2 komponentų - informacinės struktūros ir duomenų mainų posistemės.
    \begin{itemize}
        \item Informacinė struktūra. Šis komponentas atsakingas už ESPBI IS informacijos tvarkymą, saugojimą, apdorojimą ir teikimą. Pagrindinis informacinės struktūros komponentas yra elektroninės sveikatos istorijos (toliau - ESI) duomenų bazė, tačiau informacinė struktūrą sudaro ir daugiau papildomų duomenų bazių \cite{Specifikacija}. Iš viso yra 10 duomenų bazių, kurios saugo informaciją apie sveikatos priežiūros paslaugų teikimą. ESI duomenų bazėje saugomi pacientų elektroniniai sveikatos įrašai, kurie yra suvedami paslaugų ir vaizdavimo lygmenyje arba gaunami iš sveikatos priežiūros įstaigų informacinių sistemų. Kiekvienas įrašas yra susiejimas su identifikatoriumi, kuris yra suteikiamas Objektų ID katalogo. Objektų ID katalogas suteiktą identifikatorių susieja su paciento įrašu, specialistu, kuris pateikė paciento įrašą, ir sveikatos priežiūros įstaiga. ESI duomenų bazės tvarkomi dokumentai \cite{Specifikacija}: ypatingieji pacientų duomenys, ESI suformavusių sveikatinimo specialistų duomenys, ESI pateikusių sveikatinimo įstaigų duomenys;
        \item Duomenų mainų posistemė. Šis komponentas atsakingas už duomenų priėmimą bei atidavimą sveikatos priežiūros įstaigų informacinėms sistemoms bei kitoms suinteresuotų trečiųjų šalių informacinėms sistemoms. Ši posistemė yra kertinis komponentas, kuris realizuoja techninio interoperabilumo principus elektroninės sveikatos sistemoje \cite{Specifikacija}. Duomenų mainų posistemės pagrindinė paskirtis - valdyti duomenų mainus tarp sveikatos priežiūros įstaigų ir užtikrinti duomenų gavimą bei teikimą tokioms valstybinių įstaigų informacinėms sistemoms kaip SODRA, SVEIDRA, VAPRIS ir kt. Visi duomenų mainai tarp ESPBI IS ir kitų informacinių sistemų vyksta per šią posistemę.
    \end{itemize}
\end{itemize}

\subsubsection{Septinto lygio sveikatos standartas}
Ankstesniuose poskyriuose apžvelgėme Lietuvos sveikatos priežiūros įstaigų naudojamas informacines sistemas, tačiau Lietuva nėra išskirtinė šioje srityje. Kitos Europos sąjungos valstybės taip pat diegia bei naudoja tokio pat pobūdžio informacines sistemas, šių sistemų diegimo bei naudojimo būsena yra aprašyta Europos Komisijos \cite{EuroposKomisija}. Europos parlamento priimtose direktyvose \cite{EuroposParlamentas} yra nurodoma, jog valstybės turi tarpusavyje dalintis pacientų sveikatos įrašais, t.y. pacientui apsilankius svetimos valstybės sveikatos priežiūros įstaigoje, ši įstaiga galėtų gauti paciento duomenis iš jo gimtosios valstybės. Kadangi siekiama pacientų duomenų keitimosi ne tik regioniniu mastu, bet ir tarpvalstybiniu, vadinasi visos tarpusavyje komunikuojančios informacinės sistemos turi vadovautis bendru tarptautiniu standartu, kuris aprašytų pacientų duomenų apsikeitimo procesą. Toks pacientų duomenų apsikeitimo standartas yra Septintojo lygio sveikatos standartas (angl. \textit{Health Level Seven International}) (toliau - HL7). HL7 standartas, kurio pirmoji versija buvo sukurta 1987 metais, yra patvirtintas Amerikos nacionalinio standartų instituto. Šis standartas apibrėžia elektroninių sveikatos įrašų paiešką, dalijimąsi, integraciją ir apsikeitimą \cite{HL72009}. HL7 nusako kaip įstaigos keičiasi elektroniniais sveikatos įrašais ir kokiu duomenų formatu šie įrašai yra suformuoti. HL7 yra apibrėžęs 2 duomenų formatus - v2 ir v3, taip pat egzistuoja ir FHIR standartas, kuris apjungia labiausiai pasiteisinusius HL7 v2 ir v3 standartų aspektus. ESPBI IS specifikacijoje \cite{Specifikacija} yra nurodoma, jog duomenų apsikeitimui yra naudojamas HL7 v3 arba FHIR standartai, tačiau naujesnėje literatūroje \cite{Registrucentras} yra nurodomas tik FHIR. FHIR standartas apibrėžia netik duomenų formavimą, bet ir pateikia REST architektūra pagrįstą interfeisą, kuris nurodo
% paaiskint interfeisa ir resta, crud
duomenų operacijoms reikalingus CRUD metodus.

\subsection{Stacionariame gydyme pasitaikančių problemų sprendimo būdai}

Su stacionaraus gydymo problemomis, kurios buvo identifikuotos pirmąjame skyriuje, yra susiduriama ne tik Lietuvoje. Jungtinėse Amerikos Valstijose, pradėjus diegti sveikatos įstaigose informacines sistemas, atsirado daug problemų, kurios trukdė šių sistemų diegimui bei naudojimui. Viena iš problemų - sveikatos įstaigų darbuotojai nemoka naudotis informacinėmis technologijomis \cite{Jha2009}. Ši problema mažina darbuotojų produktyvumą, nes informacinių sistemų naudojimas užima perdaug laiko. Tam, kad sveikatos priežiūros įstaigų darbuotojai gebėtų efektyviai dirbti su naujausiomis technologijomis, labai svarbus yra jų įtraukimas į diegimo procesą, edukavimas ir kompetentingų žmonių pagalba \cite{Lorenzi2009}. Duomenų įvedimo sunkumai taip pat yra sprendžiami šablonų bei automatinio teksto funkcijomis \cite{Noblin2013}. Darbuotojams yra lengviau pildyti dokumentus pagal pateiktus šablonus bei instrukcijas, o automatinio teksto funkcija paspartina duomenų įvedimą į sistemą. 

Taip pat pirmąjame skyriuje buvo identifikuota medikamentų paruošimo problema. Nors daugiausiai laiko paruošimo procese užima medikamentų rūšiavimas kiekvienam pacientui, tačiau šis etapas yra gyvybiškai svarbus, nes suklydus ir pateikus netinkamą vaistą pacientui, pasekmės gali būti kritinės. Medikamentų adiministravimo problemas siūloma spręsti šiais būdais \cite{Agrawal2009}:
\begin{itemize}
    \item Paskirtų medikamentų informaciją skaitmenitizuoti;
    \item Medikamentus sužymėti brūkšninias kodais (angl. \textit{barcode}), kur brūkšniniame kode būtų laikoma paciento, kuriam paskirtas vaistas, informacija;
    \item Medikamentų užsakymo sistema (angl. \textit{Computerized physician order entry, CPOE}). Ši sistema apjungia sveikatos ir farmacijos įstaigas, padeda atlikti medikamentų užsakymus.
\end{itemize}

Pagal pateiktus problemų sprendimo pavyzdžius matome, kad dauguma jų sprendžia duomenų įvedimo problemą. Ši problema taip pat gali būti sprendžiama ALR technologija. Pagrindiniai argumentai, kodėl ALR technologija gali padėti spręsti minėtas problemas \cite{forum2}: 
\begin{itemize}
    \item Paprastas paciento duomenų perkėlimas;
    \item Užtikrintumas dėl informacijos tikslumo konkrečiam pacientui;
    \item Paprastas įrenginių prilietimas nereikalauja darbuotojų mokymų;
    \item Lengvas pacientui paskirtų medikamentų identifikavimas.
\end{itemize}