\sectionnonum{Įvadas}

\subsectionnonum{Temos aktualumas ir naujumas}
Europos Sąjungos statistikos tarnybos duomenimis 2017 metais 19\% Europos sąjungos populiacijos sudarė 65 metų ir vyresni asmenys \cite{Eurostat}. Ši tarnyba teigia, jog per paskutiniuosius 10 metų Lietuvoje 65 metų ir vyresnio amžiaus asmenų skaičius padidėjo 2,7\%, o Europos sąjungoje - 2,4\%. Populiacijos senėjimo tendencija, pasak Europos Sąjungos statistikos tarnybos, nežada keistis ir 2080 metais, 80 metų ir vyresnio amžiaus asmenų kiekis sudarys daugiau nei 12\% visos Europos sąjungos populiacijos. Pasak Pasaulio sveikatos organizacijos \cite{Organization2012}, per pastaruosius 10 metų Europos regione sveikatos priežiūros specialistų skaičius padidėjo 10\%, tačiau abejojama, ar šis specialistų skaičiaus didėjimas bus pakankamas tam, kad būtų padengtos senstančios populiacijos reikmės. Ši organizacija teigia, kad didžiausia problema, su kuria bus susiduriama, yra slaugytojų trūkumas.

Vienas iš pagrindinių valstybės sveikatinimo veiklos tikslų - užtikrinti efektyvią ir kokybišką sveikatos priežiūrą, kuri yra orientuota į gyventojų poreikius \cite{Ataskaita2018}. Sveikatos priežiūros išlaidos sudaro didelę dalį valstybės bendrojo vidaus produkto ir šios išlaidos pastaraisiais dešimtmečiais tik augo \cite{higiena}. Stacionariai sveikatos priežiūrai tenka viena iš didžiausių sveikatos priežiūros išlaidos dalių. Taip pat prognozuojama, jog stacionarios sveikatos priežiūros išlaidos ateityje nemažės \cite{higiena}. Aukščiausiosios audito institucijos atlikti auditai rodo, kad 28\% sveikatos priežiūros įstaigų, kurios teikia stacionaraus gydymo paslaugas, suteikiamų paslaugų efektyvumas yra mažas \cite{Ataskaita2018}. Kadangi stacionarioms sveikatos priežiūros paslaugoms tenka didelė dalis visų sveikatos priežiūros išlaidų ir teikiamos paslaugos nėra efektyvios bei egzistuoja tendencija, kuri rodo mažėjantį slaugytojų ir pacientų kiekio santykį, labai svarbu didinti šių paslaugų efektyvumą.

Informacinių ir komunikacinių technologijų inovacijos vis plačiau naudojamos tiek gamybos, tiek paslaugų sektoriuose. Kadangi laikoma, kad sveikatos priežiūros sektorius yra tarp gamybos ir paslaugų sektorių, minėtos inovacijos taip pat turi didelę įtaką šiam sektoriui \cite{Puma2012}. Viena iš informacinių ir komunikacinių technologijų inovacijų - aplinkos intelektas (angl. \textit{Ambient Intelligence}). Markas Weiseris, aplinkos intelekto iniciatorius, teigė, kad geriausios inovacijos yra tos, kurios tampa nebepastebimos ir susilieja su žmonių kasdieniu gyvenimu \cite{Cook2009}. Aplinkos intelektas - tai informacinių technologijų paradigma, kuria siekiama, kad kasdieniniame gyvenime naudojamos technologijos būtų įmontuotos aplinkoje, tačiau nebūtų pastebimos, o sąveika su šiomis technologijomis būtų paprasta ir patogi \cite{Bravo2008}. Sveikatos priežiūros sektoriuje, aplinkos intelektas nėra sėkmingai įgyvendintas. Sveikatos priežiūroje naudojamų informacinių technologijų sudėtingumas sunkina aplinkos intelekto įgyvendinimą, tačiau naujų technologijų panaudojimas šiame sektoriuje gali išspręsti šią problemą \cite{Fontecha2011}. Viena iš tokių technologijų yra artimojo lauko ryšio technologija (angl. \textit{Near Field Communication}). 

Artimojo lauko ryšio technologija (toliau - ALR), tai bevielio komunikavimo technologija, kurios įrenginiai komunikuoja esant mažam tarpusavio atstumui. Komunikacija tarp įrenginių vyksta elektromagnetinėmis bangomis, kurių dažnis yra 13,56 MHz \cite{Leora1980}. Ši technologija yra patogi naudojimui, nes vartotojui, kuris naudoja ALR technologija pagrįstą sistemą, užtenka priglausti ALR įrenginius ir šie įrenginiai pradeda komunikuoti. Kadangi ALR technologija pagrįstų sistemų naudojimas yra intuityvus ir pačios technologijos veikimas yra sunkiai pastebimas vartotojui, ALR technologija gali prisidėti prie aplinkos intelekto įgyvendinimo sveikatos priežiūros įstaigose. Kadangi pacientų identifikavimas yra dažnai atliekama užduotis, ALR technologija gali padėti šią užduotį atlikti efektyviau ir intuityviau. Teigiama, jog netinkamų medikamentų išdavimas yra viena iš dažniausiai įvykstančių klaidų \cite{Agrawal2009}, šiai problemai spręsti literatūrose siūlomi įvairūs būdai, vienas iš jų - ALR technologija \cite{Gautam}.



\subsectionnonum{Darbo tikslas}
Pasiūlyti ALR technologija pagrįstą programų sistemų architektūrą, didinančią stacionaraus gydymo efektyvumą.
    
\subsectionnonum{Darbo uždaviniai}
    \begin{enumerate}
        \item Išnagrinėti stacionaraus gydymo situaciją Lietuvoje:
        \begin{enumerate}
            \item Išsiaiškinti kas yra stacionarus gydymas;
            \item Apžvelgti esamą stacionaraus gydymo situaciją;
            \item Identifikuoti pagrindines stacionaraus gydymo problemas.
        \end{enumerate}
        \item Išnagrinėti informacinių technologijų taikymą Lietuvos sveikatos priežiūros įstaigose:
        \begin{enumerate}
            \item Apžvelgti ir palyginti Lietuvos sveikatos priežiūros įstaigų naudojamas informacines sistemas;
            \item Apžvelgti elektroninės sveikatos paslaugų ir bendradarbiavimo infrastruktūros informacinę sistemą ir išnagrinėti jos architektūrą, išsiaiškinti posistemių integracinius reikalavimus.
        \end{enumerate}
        \item Išsiaiškinti ALR technologijos principus:
        \begin{enumerate}
            \item Apžvelgti ALR technologijos savybes;
            \item Apžvelgti ALR technologijų taikymą sveikatos priežiūros srityje.
        \end{enumerate}
        \item Pateikti ALR technologijomis pagrįstą architektūrą:
        \begin{enumerate}
            \item Apibrėžti sistemos reikalavimus;
            \item Apžvelgti alternatyvas;
            \item Sukurti sistemos prototipą.
        \end{enumerate}
    \end{enumerate}

\subsectionnonum{Darbo struktūra}
1 skyriuje yra aprašoma stacionaraus gydymo sąvoka, pateikiamos stacionaraus gydymo paslaugų grupės. Taip pat šiame skyriuje aprašoma stacionaraus gydymo situacija Lietuvoje ir nagrinėjamos identifikuotos problemos. 2 skyriuje autorius nagrinėja skirtingų Lietuvos sveikatos priežiūros įstaigų informacines sistemas ir bendrą valstybinę įstaigų informacinę sistemą - Elektroninės sveikatos paslaugų ir bendradarbiavimo infrastruktūros informacinę sistemą. Šio skyriaus gale nagrinėjama kaip sprendžiamos 1 skyriuje identifikuotas problemos. 3 skyriuje autorius nagrinėja ALR technologijos veikimą, jos ypatybes. Šio skyriaus gale autorius aprašo ALR technologijos pritaikymą sveikatos priežiūros sektoriuje. 4 skyriuje yra siūloma sistema, aprašomi reikalavimai ir nagrinėjamos alternatyvos. Taip pat šiame skyriuje aprašomas sukurtas sistemos prototipas. Darbo pabaigoje autorius pateikia rezultatus ir išvadas.





