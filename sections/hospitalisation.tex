\section{Stacionarus gydymas}

Stacionarus gydymas, arba gydymas stacionare - tai asmens sveikatos priežiūros paslaugos. Šios paslaugos yra teikiamos sveikatos priežiūros įstaigose. Jeigu paslaugų teikimo laikas yra trumpesnis nei para, šios paslaugos yra vadinamos dienos gydymas stacionare, todėl stacionariu gydymu laikoma tokios paslaugos, kurių teikimo laikas yra ilgesnis nei para. Stacionarių paslaugų sąrašą yra patvirtinusi Sveikatos apsaugos ministerija, šį sąrašą sudaro apie 180 paslaugų \cite{StacionaroPaslaugos}. Stacionaraus gydymo paslaugos yra skirstomos į 4 grupes \cite{LigoniuKasa}: 
\begin{itemize}
    \item \textbf{Ilgalaikio gydymo paslaugos} – šios paslaugos teikiamos pacientams, kuriems yra paskirtas ilgo laikotarpio gydymas. Dažniausiai šios paslaugos teikiamos lėtinėmis ligomis sergantiems pacientams.
    \item \textbf{Transplantacijos paslaugos} – šios paslaugos teikiamos pacientams, kuriems reikalingi organų persodinimai. Į šią paslaugų grupę yra įtraukiamos tokios transplantacijos, kaip širdies, plaučių, inkstų, kaulų čiulpų ir kt. 
    \item \textbf{Aktyviojo gydymo paslaugos} – šios paslaugos teikiamos pacientams, kuriems pasireiškė lėtinių lygų pablogėjimas, atsirado agresyvios ligų formos ar patyrė sunkius sužalojimus. Teikiant šią paslaugą, pacientas yra ištiriamas, jam skiriami vaistiniai preparatai, teikiamos chirurginės paslaugos, kurios nėra teikiamos ambulatoriniame gydyme.
    \item \textbf{Medicininės reabilitacijos paslaugos} – šios paslaugos teikiamos pacientams, kuriems po sunkių būklių ar susirgimų yra būtina reabilitacija. Sunkių būklių ir susirgimų sąrašą yra patvirtintinusi Lietuvos Respublikos sveikatos apsaugos ministerija. Šią paslaugų grupę sudaro gydymas vaistiniais preparatais, gydymo dieta, fizioterapijos ir kt.
\end{itemize}

\subsection{Stacionaraus gydymo situacija Lietuvoje}

Tam, kad išsiaiškinti dabartinę Lietuvos stacionaraus gydymo situaciją, autorius pasirinko išnagrinėti stacionaraus gydymo procesus trijose Lietuvos sveikatos priežiūros įstaigose. Pagal Lietuvos statistikos departamento pateiktus duomenis \cite{Gyvento2017}, Jonavos rajono gyventojų tankis yra panašus Lietuvos vidurkiui, todėl buvo pasirinkta nagrinėti Jonavos ligoninė. Nagrinėjant šią ligoninę, buvo laikomasi nuomonės, kad šios ligoninės stacionaraus gydymo procesai bus bendriniai visų rajonų ligoninėms, išskyrus tuos rajonus, kurių gyventojų tankių rodiklius galime laikyti ekstremizmais. Prieš pradedant analizuoti stacionaraus gydymo procesus, autoriui nebuvo žinoma ar karo ligoninėse laikomasi tokių pačių procesų kaip ir civilinėse sveikatos priežiūros įstaigose. Todėl buvo pasirinkta nagrinėti vienas iš kariuomenės medicinos punktų, esantis Rukloje. Tam, kad išsiaiškinti ar egzistuoja šių procesų kritiniai skirtumai tarp mažesnių miestelių ligoninių ir didelių, buvo pasirinkta išnagrinėti vieną iš sostinės sveikatos priežiūros įstaigų - Vilniaus universiteto ligoninės Žalgirio klinika. Svarbu pažymėti, kad ne į visas sveikatos priežiūros įstaigas galima patekti asmenims, kurie neturi leidimo, todėl autorius pasirinkto, bendraujant su šių įstaigų darbuotojais, nagrinėti stacionaraus gydymo procesus ir identifikuoti šių procesų gerinimo būdus.

Nors ir aukščiau išvardintos stacionaraus gydymo paslaugų grupės skiriasi viena nuo kitos, tačiau bendrinės procedūros išlieka vienodos. Bendrinėmis stacionaraus gydymo paslaugų procedūromis galime laikyti \cite{Gautam}: 
\begin{itemize}
    \item Medikamentų suteikimas;
    \item Kūno temperatūros matavimas;
    \item Kraujospūdžio ir širdies ritmo matavimas;
    \item Medicininių tyrimų atlikimas;  
\end{itemize}
Šios procedūros yra aptinkamos visose 4 stacionaraus gydymo paslaugų grupėse, tačiau egzistuoja ne tik bendrinės procedūros, bet ir bendrinė stacionaraus gydymo eiga:
\begin{itemize}
    \item [1.] Paciento priėmimas ir lovos suteikimas;
    \item [2.] Paciento gydymo plano sudarymas;
    \item [3-5.] Paciento būklės stebėjimas, medikamentų suteikimas;
    \item [4.] Mmedicininių tyrimų paskyrimas, paciento mėginių paėmimas ir tyrimų atlikimas;
    \item [5.] Paskirtų medicininių procedūrų atlikimas;
    \item [6.] Paciento išrašymas.
\end{itemize}

Prieš paguldant pacientą, reikalingas gydymo planas, kuris yra pateikiamas paskyrimo lapo forma. Ši forma užpildoma ir patvirtinama kvalifikuoto gydytojo. Svarbu paminėti tai, jog šis planas gali būti koreguojamas paciento gydymo metu. Plane yra nurodomos reikalingos gydymo procedūros, reikalingos stebėsenos priemonės ir šių priemonių dažnumas, taip pat reikalingi medikamentai ir jų vartojimo dažnumas. Vykdant gydymo planą, kiekviena sąveika su pacientu, kuri yra nurodyta gydymo plane, yra dokumentuojama ir identifikuojamas planą vykdantis darbuotojas.
% (Parasyti apie stebejimo lapa (ar tik gydytojas ji pildo)).
\subsection{Problemos stacionariame gydyme}
2015 metais nacionaliniu mastu buvo įdiegta sveikatos priežiūros informacinė sistema, kuri leidžia visoms sveikatos priežiūros įstaigoms, taip pat ir toms, kurios neturi nuosavų informacinių sistemų, sudaryti gydymo planą bei stebėjimo lapą skaitmenizuotomis formomis, tačiau bendraujant su sveikatos priežiūros įstaigų darbuotojais paaiškėjo, jog dažnu atveju skaitmenizuotos formos nėra naudojamos arba naudojamos kaip antraeilis formų pildymas. Tokios situacijos priežastis - tiriamose įstaigose ši informacinė sistema nėra pilnai sudiegta arba jos posistemė, kuri yra skirta aptariamam gydymui, nėra patogi ir intuityvi naudojimui. Todėl gydytojui ar slaugos darbuotojui patogiau užpildyti dokumentus ranka, o atsiradus laisvai minutei, perkelti dokumentus į skaitmenitizuotą variantą. Atlikus suinteresuotos grupės apklausą, buvo pastebėta, jog skirtingų įstaigų darbuotojai susiduria su bendromis problemomis. Problemos, kurios buvo identifikuotos apklausos metu: 
\begin{itemize}
    \item Laiko eikvojimas perkeliant dokumentus į informacinę sistemą;
    \item Ilgas medikamentų paruošimo vartojimui laikas;
    \item Daug laiko sugaištama pildant popierines formas, kurios reikalingos paciento gydymo formalizavimui.
\end{itemize}

\begin{itemize}
    \item \textbf{Duomenų perkėlimas}. Nors ir nuo 2015 metų visos sveikatos priežiūros įstaigos gavo prieigą prie nacionaliniu mastu įdiegtos sveikatos paslaugų ir bendradarbiavimo infrastruktūros informacinės sistemos, tačiau naujos technologijos tik padidino sveikatos įstaigų darbuotojų darbo kiekį. Įdiegus šią sistemą, visos sveikatos priežiūros įstaigos buvo įpareigotos teikti skaitmeninius duomenis. Visi darbuotojai, dalyvavę apklausose, dirba sveikatos priežiūros įstaigose, kurios turi vidinę informacinę sistemą, todėl jie neprivalo naudotis nacionaline sistema, nes vidinė sistema komunikuoja su nacionaline. Keletas respondentų yra bandę naudotis nacionaline sistema. Visi respondentai turi panašią nuomonę - nei nacionalinė informacinė sistema, nei vidinės informacinės sistemos nėra patogios naudojimui. Respondentų nuomone, dokumentų pildymo procedūra greičiau ir efektyviau vyksta tuomet, kai viskas dokumentuojama ranka, o gydymo gale yra perkeliama į skaitmeninį formatą.
    \item \textbf{Medikamentų paruošimo laikas}. Kiekvienam pacientui gydymo plane yra paskiriami medikamentai. Vaistiniai preparatai yra užsakomi griežtai vadovaujantis gydymo planais. Gavus medikamentus stacionaraus gydymo skyriuose, šių skyrių darbuotojai atrūšiuoja medikamentus kiekvienam gydomam pacientui. Atrūšiuoti vaistiniai preparatai yra suteikiami paskirtiems pacientams, tačiau prieš pateikiant juos vartojimui, darbuotojas turi įsitikinti, kad tai išties tas medikamentas, kuris yra paskirtas. Respondentai įvardina, jog daugiausiai laiko reikalauja medikamentų rūšiavimas kiekvienam pacientui. Nors respondentai neįvardija medikamentų užsakymo kaip laiko eikvojimo, tačiau, kaip minėta anksčiau, gydymo planai yra pirmiausiai paruošiami popierine forma, o užsakant vaistinius preparatus, medikamentų duomenys yra perkeliami į skaitmeninę formą, t.y. atliekamas dvigubas darbas.
    % ar tikrai perkeliami medikamentai i kompa?
    \item \textbf{Laiko eikvojimas pildant popierines formas}. Ši problema panaši į pirmąją. Nors dauguma respondentų mano, kad jiems yra greičiau užpildyti visus reikiamus dokumentus popierine forma, o atsiradus laikui, dokumentus perkelti į  informacinę sistemą, tačiau pripažįsta, kad popierinių dokumentų pildymas nėra efektyvus ir užima nemažai laiko sąnaudų.
\end{itemize}

Taip pat svarbu pažymėti, jog, apklausos metu, respondentai minėjo susiduriantys su pacientų bandymu įduoti kyšį. Pacientai, kurie bando duoti kyšį, grindžia savo veiksmus tuo, kad darbuotojas, priėmęs kyšį, pacientui suteiks kokybiškesnį gydymą ir priežiūrą, arba bent jau ne prastesnį nei kitiems pacientams. Respondentų minima problema tik patvirtina Valstybinės ligonių kasos darytos apklausos rezultatus \cite{Kasa2016}. Apklausos rezultatai parodė, jog 65\% respondentų mano, jog gydymo ir priežiūros kokybė priklauso nuo kyšio davimo. 

% Nors ši problema nėra klasifikuojama kartu su stacionaraus gydymo procesų problemomis, tačiau šiame darbe bus siūlomi ALR technologija paremti sprendimai, kurie padėtų keisti pacientų nuomonę dėl kyšio davimo ir spręstų išvardintas problemas.





% jei atvyksta i priemima, jis rusiuojamas, pagal sunkuma ligos. Jei su greitaja, tai tiesiai i renimacija. Sutinka sesute ir daktras, ji prima, uzregistruoajam, uzveda forma, iveda i sistema, sumatuoja temp, pulsa, kraujospudi, uzraso kardiograma, priskiria prioriteta ir jis laukia gydytojo apziurojs. Yra formos tam tikros, jas daktaras uzpildo formas, jas atspausdina. Kiekvienas gydytojas palieka savo israsus. 

% Priemime uzveda ligos istorija, joje statusas (pvz: bukle tapati, ar nepablogejus, o jei kazkas atsisitiko skiriami nauji vaistai) turbi buti anspaudas gydytojo. I eina i ta blanka kelios formos, PASKIRIMO lapas, ji uzpildo gydytojas, koreguoja irgi gydytojas, deda parasa. TEn isladinti 3kartus dienoj kazka (laseline, vaistai). Stebejimo lape - statusa nurodo ligonio.

% Dienos stacionare teikiamos nėštumo patologijos, dermatovenerologijos, onkohematologijos, hematologijos, terapijos, vaikų ligų, alerginių ligų, spindulinės terapijos ir kt. paslaugos

% Skaitykite daugiau: https://sveikata.lrytas.lt/sveikas/2016/03/22/news/kas-yra-dienos-stacionaro-paslaugos--1517839/?utm_source=lrExtraLinks&utm_campaign=Copy&utm_medium=Copy