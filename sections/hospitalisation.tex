\section{Stacionarus gydymas}

Stacionarus gydymas, arba gydymas stacionare - tai asmens sveikatos priežiūros paslaugos. Šios paslaugos yra teikiamos stacionarinėse sveikatos priežiūros įstaigose. Jeigu paslaugų teikimo laikas yra trumpesnis nei para - tai dienos gydymas stacionare, todėl stacionariu gydymu laikoma tada, kai paslaugų teikimo laikas yra ilgesnis nei para. Stacionarių paslaugų sarašą yra patvirtinusi Sveikatos apsaugos ministerija, šį sąrašą sudaro apie 180 paslaugų \cite{StacionaroPaslaugos}. Stacionaraus gydymo paslaugos yra skirstomos į \cite{LigoniuKasa}: 
\begin{itemize}
    \item \textbf{Ilgalaikio gydymo paslaugos} – šios paslaugos teikiamos pacientams, kuriems yra paskirtas ilgo laikotarpio gydymas. Šios paslaugos reikalingos pacientams, kurie serga lėtinėmis ligomis.
    \item \textbf{Transplantacijos paslaugos} – šios paslaugos teikiamos pacientams, kuriem reikalingi organų persodinimai. Šiose paslaugos įtraukiamos tokios transplantacijos kaip - širdies, plaučių, inkstų, kaulų čiulpų ir kt. 
    \item \textbf{Aktyviojo gydymo paslaugos} – šios paslaugos teikamos pacientams, kuriems pasireiškė lėtinių lygų pablogėjimas, atsirado agresyvios ligų formos, patyrė sunkios sužalojimus. Teikiant šį gydymą, pacientas yra ištyriamas, jam skiriami vaistiniai preparatai, teikiamas chirurginės paslaugos, kurios neteikiamos ambulatoriniame gydyme.
    \item \textbf{Medicininės reabilitacijos paslaugos} – šios paslaugos teikiamos pacientams, kuriems po sunkių buklių ar susirgumų, kurie yra patvirtinti Lietuvos Respublikos sveikatos apsaugos ministerijos, yra teikiama reabilitacija. Į šią paslaugą įeina gydymas vaistiniai preparatais, gydymo dieta, fizioterapijos ir kt.
\end{itemize}
% Inpatient care refers to medical treatment that is provided in a hospital or other facility and requires at least one overnight stay.

% For example, hospitalists are physicians who practice only inpatient care, and no office-based or outpatient care.

% Inpatient care tends to be directed towards more serious ailments and trauma that require one or more days of overnight stay at a hospital. For the purposes of healthcare coverage, health insurance plans require you to be formally admitted to a hospital for a stay for a service to be considered inpatient. This means a doctor has to write a note to give the order to admit you, so if you were in the emergency room and were asked to stay overnight for “Medical Observation”, it does not make you an inpatient.



% jei atvyksta i priemima, jis rusiuojamas, pagal sunkuma ligos. Jei su greitaja, tai tiesiai i renimacija. Sutinka sesute ir daktras, ji prima, uzregistruoajam, uzveda forma, iveda i sistema, sumatuoja temp, pulsa, kraujospudi, uzraso kardiograma, priskiria prioriteta ir jis laukia gydytojo apziurojs. Yra formos tam tikros, jas daktaras uzpildo formas, jas atspausdina. Kiekvienas gydytojas palieka savo israsus. 

% Priemime uzveda ligos istorija, joje statusas (pvz: bukle tapati, ar nepablogejus, o jei kazkas atsisitiko skiriami nauji vaistai) turbi buti anspaudas gydytojo. I eina i ta blanka kelios formos, PASKIRIMO lapas, ji uzpildo gydytojas, koreguoja irgi gydytojas, deda parasa. TEn isladinti 3kartus dienoj kazka (laseline, vaistai). Stebejimo lape - statusa nurodo ligonio.

% Dienos stacionare teikiamos nėštumo patologijos, dermatovenerologijos, onkohematologijos, hematologijos, terapijos, vaikų ligų, alerginių ligų, spindulinės terapijos ir kt. paslaugos

% Skaitykite daugiau: https://sveikata.lrytas.lt/sveikas/2016/03/22/news/kas-yra-dienos-stacionaro-paslaugos--1517839/?utm_source=lrExtraLinks&utm_campaign=Copy&utm_medium=Copy