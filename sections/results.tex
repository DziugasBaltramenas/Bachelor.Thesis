\sectionnonum{Rezultatai ir išvados}
Šiame darbe išanalizuota stacionaraus gydymo situacija Lietuvoje, identifikuotos problemos, o kartu ir apžvelgiami šių problemų sprendimo būdai. Taip pat darbe aprašytos ir išnagrinėtos kelios sveikatos priežiūros įstaigų informacinės sistemos. Nagrinėjant ALR technologiją, išsiaiškinta, kad ši technologija yra tinkama stacionariame gydyme teikiamų paslaugų efektyvumo didinimui. Šiame darbe yra siūloma programų sistemų architektūra, kuri yra pagrįsta ALR technologija, ir pateikiamas jos prototipas.

Atlikus darbą buvo gauti šie \textbf{rezultatai}:
\begin{enumerate}
    \item Išsiaiškinta kas yra stacionarus gydymas, kokia jo situacija Lietuvoje ir identifikuotos stacionariame gydyme kylančios problemos bei nustatyti šių problemų sprendimo būdai;
    \item Atliktas Lietuvos sveiktos priežiūros įstaigų informacinių sistemų palyginimas, išanalizuota elektroninės sveikatos paslaugų ir bendradarbiavimo infrastruktūros informacinė sistema ir nustatyti integraciniai reikalavimai;
    \item Remiantis literatūra, išanalizuotas ALR technologijos veikimas, nustatytos pritaikymo galimybės sveikatos priežiūroje;
    \item Iškelti programų sistemų architektūros funkciniai ir nefunkciniai reikalavimai, analizės būdu nustatyta ir pasiūlyta architektūra, sukurtas sistemos prototipas.
\end{enumerate}

Darbo \textbf{išvados}:
\begin{enumerate}
    \item Stacionariame gydyme teikiamų paslaugų efektyvumas nėra aukštas, tačiau ALR technologija gali prisidėti prie sėkmingo aplinkos intelekto įgyvendinimo sveikatos priežiūros sektoriuje ir didinti stacionariame gydyme teikiamų paslaugų efektyvumą;
    \item Ne visos sveikatos priežiūros įstaigos turi vidines informacines sistemas. Tačiau įstaigų, turinčių šias sistemas, informacinės sistemos nėra vienodos, jų sudėtingumo lygiai ir teikiami funkcionalumai skiriasi. Informacinių sistemų sudėtingumas priklauso nuo įstaigos dydžio;
    \item Norint įdiegti naują technologiją į sveikatos priežiūros sektorių, dėl įstaigų informacinių sistemų nevienodumo, verta kurti atskirą informacinę sistemą ir atlikti integraciją su esamomis informacinėmis sistemomis;
    \item Sveikatos priežiūros įstaigų naudojamas informacines sistemas yra svarbu kurti taip, jog jos būtų paprastos naudojimui, nesudėtingai plečiamos ir užtikrintų pacientų duomenų saugumą;
    \item Siūlomos sistemos įgyvendinimo kaštai kiekvienai sveikatos priežiūros įstaigai skiriasi. Tai priklauso nuo to, ar įstaiga turi savo vidinę informacinę sistemą, kuri yra integruota į elektroninės sveikatos paslaugų ir bendradarbiavimo infrastruktūros informacinę sistemą, ar tokios sistemos neturi;
    \item Įgyvendinus ir atliktus siūlomos sistemos prototipo testavimą, galima teigti, kad siūloma ALR technologija paremta sistema, kuri padeda atlikti efektyviau bendrines stacionaraus gydymo procedūras, yra įgyvendinama.
\end{enumerate}

Viską apibendrinus galima teigti, kad išsikeltas darbo tikslas buvo pasiektas. Išnagrinėta Lietuvos stacionaraus gydymo situacija, identifikuotos problemos, apžvelgti šių problemų sprendimo būdai ir analizuojant alternatyvas, pasiūlytas programų sistemų architektūros sprendimas, kuris didina stacionaraus gydymo efektyvumą.

Ateityje šį darbą galima tęsti šiais aspektais:
\begin{enumerate}
    \item Sukurti pilnai veikiantį siūlomos programų sistemos architektūros sprendimą ir integruoti į pasirinktą sveikatos priežiūros įstaigą;
    \item Išanalizuoti ir atlikti žmogaus gyvybinių požymių sensorių diegimą į siūlomą sistemą.
\end{enumerate}