% \section{Notes}

% \subsection{Jonavos ligonine}
% zmogus, uzsiregistruoja, (kklinikuose -priemimo registratura (priemimo langelis)), paduodi dokumenta, kas nutiko ir uzveda reikalingus dokuemntas(ligoniu kasuose parasyta kokie dokai turi but pildomi). Jeigu primemei - statistine forma, kur eina ligoniu kasom, pagal kuria mokama uz paslauga. Nurodoma kas etveze( ar pats atejo ar atveze). Laukia pacientas kol pakvies. Yra stebejimo eiles, kurias skirsto pagal sunkuma. Daktaras apziuri pacienta, jam paskriia tyrimus, instrumentini tyrima (viskas vyksta e sistemoje) paima krauja, jei jo reikia.


% Stebejimo lapai. Nukle nuolat stebima, jame nurodoma kaip stebimas, ar gereja zmogui (stacionarizuoti) - kita forma ir guldo i ligonine. Paskiria lova, guldo i palata. Kai paleidzia namo - pildomas klinikinis irasas, apraso visa amneze, kas buvo padaryta.

% priemime 025 forma(ambulatorinis)
% stavionare 066 forma
% isleidziant - apikrize galutine

% Not only can NFC tags provide medical professionals with information about what treatments a patient should receive, but they can also keep track of when nurses and doctors have checked in with that patient and when. Each time the tag is scanned, the information about who scanned it and when can be transferred to a database. In addition to improving treatment, NFC tags also have potential in the research realm.


% E-health card system consist from 5 subsystems −Electronic  health  card  system  of  citizens  –  is  used  by  Maternity  Hospitals,  Child  Policlinics  and  Child Hospitals −Medical  checkup  card  system  –  is  used  by  Sanitary  &  Epidemiological  Centers  and  medical  checkup cabinets −Information  Management  Center  –  is  used  to  manage  system  users,  rights,  to  monitoring  system  usage, supervising healthcare services −Card  Personalization  Center  –  is  used  to  personalize  e-health  cards,  medical  checkup  cards  and  professional cards. −e-health portal – enables citizens securely access to all the information about himself stored in the system.





% https://www.himss.org/library/benefits-and-barriers-rfid-technology-healthcare

% As a radio frequency identification technology (RFID), an NFC tag can carry more data than a barcode or QR code and provide access to a virtually unlimited realm of additional information. As a result, the currently small market for NFC technology is growing rapidly with a compound annual growth rate of 37% projected between 2015 and 2020, according to a report from IDTechEx (1).
% R. Das and P. Harrop, “RFID Forecasts, Players and Opportunities 2016-2026,” IDTechEx Market Study, October 2015.



% galima pritempti prie kitu nfc panaudojimu medecinoje
% https://www.nfcworld.com/2011/09/06/39716/50000-dutch-nurses-now-using-nfc-phones/
% cia galima cituoti
% https://www.ncbi.nlm.nih.gov/pmc/articles/PMC5334128/ 

% jei atvyksta i priemima, jis rusiuojamas, pagal sunkuma ligos. Jei su greitaja, tai tiesiai i renimacija. Sutinka sesute ir daktras, ji prima, uzregistruoajam, uzveda forma, iveda i sistema, sumatuoja temp, pulsa, kraujospudi, uzraso kardiograma, priskiria prioriteta ir jis laukia gydytojo apziurojs. Yra formos tam tikros, jas daktaras uzpildo formas, jas atspausdina. Kiekvienas gydytojas palieka savo israsus. 

% Priemime uzveda ligos istorija, joje statusas (pvz: bukle tapati, ar nepablogejus, o jei kazkas atsisitiko skiriami nauji vaistai) turbi buti anspaudas gydytojo. I eina i ta blanka kelios formos, PASKIRIMO lapas, ji uzpildo gydytojas, koreguoja irgi gydytojas, deda parasa. TEn isladinti 3kartus dienoj kazka (laseline, vaistai). Stebejimo lape - statusa nurodo ligonio.

% Tvarka priklauso nuo skyriaus ( chirurginiam kitokios gali buti).


% 1) aprasom stacionaru gydyma
% 2) aprasom dabartine sistema 
% 3) aprasom tecnologijas taikomas medicinoj pasaulio






pastabos:

ismesti is tikslo protipoa ir kastus.



pasigilinti identifikuotas problemas
preizastis kodel jis kyla


mazinti nfc technolgoijos aprasa galbut






1)kaip surenkami reikalavimai architekturos, sistemos

(nelaiku ivedami duomenys)


pleciamumas,  plesti gali ligonine pradeti teikti naujas paslugas

atstotuma apgalvoti
galbut nasumas sistemos


ar tisklas kurti nauja sistem,a ar kuti architetura kuri apvelka visas sistemas





vieni kyla is problemu, kiti kas privaloma is dalykines srities

gali atsirasti nfc techniniai ribojimai, ju sprendimam kyla reikalavimai




nefunkc reikalavimai bendri dazniausiai
atstotumumas, 


2)architekturos alternativos, kaip jas sugalvoti, gal yra pavyzdziu?





PRAKTIKA

1) ka reiketu akcentuoti

parodyti, kad atlikom praktika, parasyt ka as padariau. Ka as padariau, o ne ka kiti

ka man reikejo padaryt, kokius sprendimus reikejo padarryt, ka issiaiskint 

apsitarti ar galiu mineti bba training




DUOMENU IVEDIMAS mano pagrd funkc


_________________________________________________________________________________________________________________________________________________________________________________________________________________

NAUJAS MEETAS


ivade butinai reikia pamineti medicina ir nfc, duhh :DDD

reik susieti nfc su medicina, vienas is budu galetu buti

turetu 2.3 poskyris, stacionariame gydyme pasitaikanciu sprendimo budai, butu apzvelgiama kaip db yra sprendziama dar galima spresti per nfc (reciau panaudojimas bet galima sprendimas -nfc)



reikalavimai - iskur atsiranda reikalavimas ( nurodyt kur nagrineta buvo). 


kurie reikalavimai sako reik nfc panaudot?


sistema turi turetu supaprastintus duomenu ivedimo budus.



tada seka architekruros aprasymas, pirmiausia vizija sugalvoti
1. egzistuojanciu architekturu pritaikymu galimybes, sistemoje, kuriose naudojamos nfc, (kas tinka, o ka reikia keisti) papildomas lygmuo -valdikliu;
visom sistemo reiketu pritaikyti mano sistema

kam nauja irengini prijungti (nfc) (seku diagrama kurt) kas atsitinka jei neveikia

jei integruotjam su bendraja sistema


sistemos pleciamuma paziureti, kokie pletimo taskai? (tarkim kitokios strukturos duomenis??) (ar kitokie nfc chipai bus naudojami?)


4+1 architekturos modelis
svarbus: informacinis vaizdas(duomenu strukturas nupaiso, pati paskutine bus, alternatyvu nereik), fizinis galbut, scenariju vaizdas( bet ne visus, bet unikalius ,kurie atsiranda del nfc ivedimo), 



prototipo aprasymas reiks itraukt kaip buvo isbandyta




